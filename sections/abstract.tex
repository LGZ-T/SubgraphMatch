\begin{abstract}
Subgraph search finds all subgraphs of a data graph that are isomorphic to a query graph. It is a fundamental operation in many application fields like analysis of protein-protein interaction network and community detection. Existing works adopt a simple search procedure that matches vertices of a query graph one by one, which incurs a large number of memory accesses during reading and writing intermediate results. In this work, we perform subGraph sEarch usiNg parallEl Vertex mAtching (\SystemName). Specifically, \SystemName can match as many query vertices as possible at each iteration and generate corresponding results in one GPU kernel. Compared to GSI, which is the state-of-the-art GPU-based subgraph search method, our approach achieves an average speedup of $5\times$. Additionally, we optimize the data graph format of GSI by replacing its hash indexes with interval indexes. The proposed interval-index format reduces the space cost and searching time of hash-index format by 83\% and 58\% respectively. To validate the effectiveness of parallel vertex matching, we also compare it to the single vertex matching devised based on our  parallel vertex matching. Results show that our approach improves the single vertex matching by 15.9\% on average.

%Based on properties of the data graph, there are many types of subgraph matching. In this work, we focus on matching query graphs on a labeled undirected data graph with the acceleration of GPU.
\end{abstract}
