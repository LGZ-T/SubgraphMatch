\section{Introduction}
%Various areas, including social networks, chemical compounds, graph neural networks, and citation networks, use graphs as their underlying data structure. With the widespread adoption of graphs and the increasing graph size, many algorithms have been proposed to analyze graphs efficiently. Among these algorithms, subgraph search has always played an essential role in graph data mining.
%
%The goal of subgraph search is to find all subgraphs in a data graph $G$ that are isomorphic to a given query graph $q$. Though the problem
%is NP-complete, many approaches
%\cite{bhattarai2019ceci,guo2020gpu,tran2015fast,shi2020graphpi,bi2016efficient,zeng2020gsi,sun2020subgraph,guo2020exploiting,sun2020rapidmatch,lin2016network}
%have been proposed to speed up subgraph search in recent years. All these approaches adopt a similar 3-step procedure. They first build
%candidate sets and auxiliary data for query vertices, then generate a matching order for query vertices based on candidate sets and
%auxiliary data and finally match $q$ in $G$ according to the matching order. Most previous works mainly focus on generating an effective
%matching order that can reduce the number of intermediate results during subgraph search. For example, the main idea of the matching order
%generation algorithm proposed in \cite{bi2016efficient} is to match all non-tree edges regarding any spanning tree of $q$ as soon as
%possible to eliminate invalid subgraph candidates in the early stages of the matching process. We borrow this idea when designing our
%matching order generation algorithm.
%
%Some works \cite{lin2016network,guo2020gpu,tran2015fast,zeng2020gsi,guo2020exploiting} focus on optimizing subgraph search on GPUs, and GSI
%\cite{zeng2020gsi} achieves the best performance among all these approaches. GSI designs an edge label partitioned CSR (PCSR) format for a
%data graph to speed up accessing vertex neighbors. To build a PCSR format for a data graph, GSI first groups edges and corresponding
%vertices by the edge label into edge label partitions. Then, GSI builds a GPU-based CSR format for each edge label partition, which is the
%PCSR format. In order to find the position of a given vertex ID (VID) in PCSR, GSI adopts a hash function to map VIDs to positions
%(hash-PCSR), which needs many empty entries to reduce collisions (GSI uses 30 empty entries for each VID). In our approach, we also utilize
%PCSR format but replace the hash function with interval indexes (interval-PCSR). Additionally, we design a VID mapping algorithm for
%vertices in $G$ to map old VIDs to new VIDs, which can create more contiguous VIDs in each edge label partition and hence reduce the number
%of intervals. GSI proposes a Prealloc-Combine approach to make the parallel write of intermediate results more efficient. Nevertheless,
%this approach needs to launch two extra GPU kernels to write intermediate results to the right place. Unlike GSI, we utilize atomic
%operations inside the GPU kernel to calculate right positions for intermediate results. Therefore, we do not need extra GPU kernels.

Subgraph matching is a fundamental task of graph analysis. It requires finding all subgraphs from a data graph $G$ that are isomorphic to a
query graph $q$. Subgraph matching requires finding all the isomorphic subgraphs (known as graph embeddings) from the data graph $G$ by
ensuring both the vertices and the edge labels of the extract subgraph maths the vertices and edge label of the query graph $q$. This
technique has a wide range of applications, including social network analysis \cite{wang2012truss,kairam2012The}, and chemical compound
search \cite{wooyoung2011Biological}.

Subgraph matching is an NP-complete problem \cite{garey1979Computers}, requiring significant computation time when processing large, real-life
graphs. A wide range of approaches have been proposed to speed up subgraph search
\cite{bhattarai2019ceci,guo2020gpu,tran2015fast,shi2020graphpi,bi2016efficient,zeng2020gsi,sun2020subgraph,guo2020exploiting,sun2020rapidmatch,lin2016network}.
These approaches adopt a typical 3-step process for subgraph matching. For the vertices in the query graph $q$, this process first builds
candidate sets and auxiliary data from the data graph $G$. It then determines a matching order that determines how each query vertex
should be matched from the candidate sets and auxiliary data before performing the graph matching by following the matching order.

Some of the most recent works attempt to leverage the GPU computation power for fast graph matching
\cite{lin2016network,guo2020gpu,tran2015fast,zeng2020gsi,guo2020exploiting}. GSI is the current state-of-the-art GPU-based graph matching
algorithm \cite{zeng2020gsi}, delivering the best performance on some of the representative datasets. It introduces a dedicated graph
storage format to reduce the memory footprint and improve the performance for graph vertex matching. While promising, existing approaches
can only match one vertex from the query graph in one go. Such a strategy leads to extensive GPU memory accesses because they need to load
and store the intermediate results for each vertex matching. These memory operations are expensive on GPUs with sizeable intermediate
results, which should be avoided if possible.


Most subgraph matching approaches do not explore data parallelism within a parallel thread because each thread only matches one vertex from
the query graph in each iteration. Such a strategy leads to extensive GPU memory accesses because the GPU worker needs to load and store
the intermediate results for each matched vertex. As we will show later in the paper, these memory operations are expensive on GPUs with
sizeable intermediate results, leaving much room for performance improvement. While there are techniques for reducing the intermediate
results on multi-core CPUs by simultaneously processing multiple graph edges \cite{lai2015scalable}, their strategy can handle a small set
of vertice matching patterns. Our work aims to close this gap by enabling the simultaneous process of multiple vertices in one go to reduce
the GPU memory accesses.

We propose \SystemName\footnote{\SystemName = subGraph sEarch usiNg parallEl Vertex mAtching.}, a new GPU-based subgraph matching framework
for parallel vertex matching within a GPU worker. \SystemName aims to match multiple vertices at each iteration and produce the
corresponding results in one GPU kernel. Unlike prior work \cite{lai2015scalable}, our approach support seven representative matching
patterns (Figure \ref{fig:matchpattern}). It uses one single GPU kernel to match multiple edges in a single GPU kernel simultaneously. By
matching multiple vertices and edges in a single GPU kernel, our approach eliminates the expensive GPU global memory addresses. \SystemName
provides new optimizing algorithms to generate the vertex matching order and process the commonly used matching pattern. \SystemName also
introduces a new graph structure storage format, which gives significant benefits for graph data storage overhead and processing time over
GSI.

We evaluate \SystemName by applying it to eight real-world data graphs on an NVIDIA 2080ti GPU. We compare \SystemName against GSI, the
state-of-the-art GPU-based subgraph matching framework. Experimental results show that \SystemName reduces the processing time by $5\times$
on average over GSI. It uses 83\% less memory for graph data storage and improves the vertex search time by 58%.




%All the abovementioned works match vertices of $q$ one by one, and thus need to write the intermediate results after matching one vertex
%and read the same intermediate results before matching the next vertex. The write and read operations are time-consuming, especially when
%the size of intermediate results is significant. To mitigate the cost of write and read operations, Lai et al. \cite{lai2015scalable}
%utilize MapReduce to implement a double-edge extension method. Their approach iteratively matches the query graph by a TwinTwig that
%consists of one edge (matching pattern 0 in Figure \ref{fig:matchpattern}) or two incident edges of a vertex (matching patterns 1 and 3 in
%Figure \ref{fig:matchpattern}). However, there are three inherent limitations in \cite{lai2015scalable}. First, edges in a TwinTwig must
%grow from the same vertex, which means their approach can not handle matching patterns 2, 4, and 7 in Figure \ref{fig:matchpattern}.
%Second, a TwinTwig contains at most two edges, but more than two edges can be processed at the same time, like matching patterns 5 and 6.
%Third, they use a simple method to generate results for matching pattern 1, which causes a number of unnecessary memory accesses to
%neighbors.

%To overcome limitations of \cite{lai2015scalable}, we propose an approach that performs subGraph sEarch usiNg parallEl Vertex mAtching
%(\SystemName).  Specifically, \SystemName matches as many query vertices as possible at each iteration and generate corresponding results
%in one GPU kernel. First, our approach can match vertices from different source vertices, e.g., matching patterns 2 and 4. Second, our
%approach can match as many edges as possible in a single GPU kernel (illustrated in Section \ref{sec:eliphase}). Third, we design an
%optimized algorithm (Algorithm \ref{algo:optDV}) to reduce unnecessary memory accesses for matching pattern 1. Moreover, we develop a new
%matching order generation algorithm (Algorithm \ref{algo:genmatchorder}) that is designed specifically for \SystemName. In summary, our
%approach can handle all matching patterns in Figure \ref{fig:matchpattern}.


This paper makes the following technical contributions:
 \begin{itemize}
\item It presents a novel parallel vertex matching method to support the process of multiple query vertices at the same time to reduce
    the GPU global memory access operations (Section \ref {sec:overview});
\item It presents an enhanced storage format to improve the storage efficiency and reduce the vertex search time (Section
    \ref{sec:storage});
\item It introduces an enhanced matching order generation algorithm to produce an appropriate vertex matching order to support efficient
    subgraph matching (Section \ref {sec:matchingorder}).
\end{itemize}


% We validate our approach from three aspects: (1) We obtain an average speedup of $5\times$ over GSI for subgraph search; (2) Experimental
% results show that our enhanced storage format significantly reduces the space cost and searching time of the storage format of GSI by 83\%
% and 58\% respectively; (3) Compared to the single vertex matching , which is modified based on our parallel vertex matching, our approach
% reduces the search time by 15.9\% on average.

% To summarize, we make the following contributions:
% \begin{itemize}
%    \item We propose a parallel vertex matching method that can match as many query vertices as possible to reduce the number of read and write operations of intermediate results.
%    \item We propose an enhanced storage format and a vertex ID mapping algorithm, which significantly reduces the space cost and searching time of the storage format of GSI.
%    \item We propose a new matching order generation algorithm and a new parallel write scheme to accommodate \SystemName.
%\end{itemize}
